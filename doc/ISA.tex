% ISA Dokumentation 
	% Beispiel MIPS https://s3-eu-west-1.amazonaws.com/downloads-mips/documents/MD00082-2B-MIPS32INT-AFP-06.01.pdf
	% alle Instruktionen die eine Schnittstelle zwischen Programmierer und Prozessor bilden
	% ggf. auf CPU Programming Model eingehen
		% vllt. erstmal zu viel vorweggegriffen
		% ggf. Register vorstellen
		% ggf. auf Memory Alignment eingehen
		% noch weitere Punkte
	% divide into functional groups
		% bei uns vllt. gemäß InstrType
	% Instruktionen generell beschreiben
		% vgl. 5.1 CPU Load and Store Instructions
		% ggf. noch nicht zu ausführlich, falls sich da noch Dinge ändern
		% vllt. von allen Typen jeweils am 1. Absatz orientieren
	% Instruktionen auflisten
		% tabellarisch https://de.overleaf.com/learn/latex/Tables
		% mit Name, Operanden, Effekt
		
\documentclass{article}

\usepackage{multirow}
\usepackage{amssymb}

\begin{document}
{
\centering
\begin{tabular}{ |p{2cm}||p{2cm}|p{3cm}|p{5cm}|  }
 \hline
 \multicolumn{4}{|c|}{Instructions} \\
 \hline %hier weiter
 Instr. Kind&Instr. Name&Operators&Description\\
 \hline
\multirow{8}{2cm}{ALU Instructions without Immediate} & add &Reg1, Reg2, Reg3&   Reg1 $:=$ Reg2 $+$ Reg3\\
&sub& Reg1, Reg2, Reg3   & Reg1 $:=$ Reg2 $-$ Reg3\\
&sll& Reg1, Reg2, Reg3&  Reg1 $:=$ Reg2 $<<$ Reg3\\
&srl& Reg1, Reg2, Reg3&  Reg1 $:=$ Reg2 $>>$ Reg3 logical (fill space with 0)\\
&sra  & Reg1, Reg2, Reg3& Reg1 $:=$ Reg2 $>>$ Reg3 arithmetical (fill space with sign bit)\\
&xor& Reg1, Reg2, Reg3 &  Reg1 $:=$ Reg2 xor Reg3\\
&or& Reg1, Reg2, Reg3&  Reg1 $:=$ Reg2 or Reg3\\
&and& Reg1, Reg2, Reg3&  Reg1 $:=$ Reg2 and Reg3\\
 \hline
 \multirow{8}{2cm}{ALU Instructions with Immediate} & addi &Reg1, Reg2, Imm&  Reg1 $:=$ Reg2 $+$ Imm\\
&subi& Reg1, Reg2, Imm   & Reg1 $:=$ Reg2 $-$ Imm\\
&slli& Reg1, Reg2, Imm&  Reg1 $:=$ Reg2 $<<$ Imm\\
&srli& Reg1, Reg2, Imm&  Reg1 $:=$ Reg2 $>>$ Imm logical (fill space with 0)\\
&srai  & Reg1, Reg2, Imm&Reg1 $:=$ Reg2 $>>$ Imm arithmetical (fill space with sign bit)\\
&xori& Reg1, Reg2, Imm&Reg1 $:=$ Reg2 xor Imm\\
&ori& Reg1, Reg2, Imm&Reg1 $:=$ Reg2 or Imm\\
&andi& Reg1, Reg2, Imm&Reg1 $:=$ Reg2 and Imm\\
 \hline
 %sources 1, 2 destination 0
  \multirow{5}{2cm}{Memory Instructions} & lw &Reg1, Reg2, Imm&addr$:=$Reg2(base)$+$Imm(offset), Reg1$:=$(Mem\_word[addr])\\
&lb& Reg1, Reg2, Imm   &addr$:=$Reg2(base)$+$Imm(offset), Reg1$:=$(Mem\_byte[addr])\\
&sw& Reg1, Reg2, Imm  &addr$:=$Reg2(base)$+$Imm(offset), Mem\_word[addr]$:=$Reg1\\
&sb& Reg1, Reg2, Imm  &addr$:=$Reg2(base)$+$Imm(offset), Mem\_byte[addr]$:=$Reg1\\
%sources 0,1
&flush  & Reg, Imm&addr$:=$Reg(base)$+$Imm(offset), flush\_cashline(adr)\\
 \hline
 %source 0, 1, 2
  \multirow{8}{2cm}{Branch Instructions} & beq &Reg1, Reg2, Label& pc$:=$Label if Reg1$==$Reg2, else pc$+=$1\\
&bne& Reg1, Reg2, Label&pc$:=$Label if Reg1$!=$Reg2\\
&bltu& Reg1, Reg2, Label&  pc$:=$Label if u(Reg1)$<$u(Reg2)\\
&bleu& Reg1, Reg2, Label&  pc$:=$Label if u(Reg1)$<=$u(Reg2)\\
&bgtu  & Reg1, Reg2, Label&pc$:=$Label if u(Reg1)$>$u(Reg2)\\
&bgeu& Reg1, Reg2, Label&pc$:=$Label if u(Reg1)$>=$u(Reg2)\\
&blts& Reg1, Reg2, Label&pc$:=$Label if s(Reg1)$<$s(Reg2)\\
&bles& Reg1, Reg2, Label&pc$:=$Label if s(Reg1)$<=$s(Reg2)\\
&bgts& Reg1, Reg2, Label&pc$:=$Label if s(Reg1)$>$s(Reg2)\\
&bges& Reg1, Reg2, Label&pc$:=$Label if s(Reg1)$>=$s(Reg2)\\
 \hline
   \multirow{2}{2cm}{Special Instructions} & rdtsc &Reg& Reg$:=$cyclecount(number of executed execution unit ticks)\\
&fence&none&all instr in the EU unit at the point of issueing the fence are executed before the fence ist executed; no new instructions are issued before the fence is executed\\
 \hline
\end{tabular}
}
\end{document}